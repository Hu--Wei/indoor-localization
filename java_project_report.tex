\documentclass[a4paper,10pt]{article}
\usepackage[utf8]{inputenc}
\usepackage{CJK, CJKnumb}
\usepackage{amsmath,amsfonts,amsthm,amssymb}
\usepackage{setspace}
\usepackage{fancyhdr}
\usepackage{lastpage}
\usepackage{extramarks}
\usepackage{chngpage}
\usepackage{soul,color}
\usepackage{xcolor}
\usepackage{graphicx,float,wrapfig}
\usepackage{listings}
\usepackage{enumerate}
\usepackage{multirow}
\usepackage{pxfonts}
\renewcommand{\today}{\number\year 年 \number\month 月 \number\day 日}

\topmargin=-0.45in      %
\evensidemargin=0in     %
\oddsidemargin=0in      %
\textwidth=6.5in        %
\textheight=9.0in       %
\headsep=0.25in         %

\begin{document}

\lstset{numbers = left,
numberstyle = \tiny,
frame = shadowbox,
rulesepcolor = \color{red!20!green!20!blue!20},
tabsize = 4,
language=Java,
basicstyle=\tt
}
\begin{CJK*}{UTF8}{gbsn}
%opening
\CJKtilde
\title{Indoor Localization设计报告}
\author{}
\date{}
\maketitle

\section{程序功能}
Indoor Localization主要实现了查询位置和上传Wifi数据两大功能。
\begin{enumerate}
 \item 查询位置时,只需要点击``查询''按钮,应用就会开始扫描当前位置的Wifi信号,然后将数据上传到服务器。服务器会返回预测到的房间号,以文本形式展示在屏幕上。
 同时在地图上会显示服务器返回的位置信息和用户当前面向的方向。整个过程大约需要3到4秒时间,过程中会有进度条指示进度,并且也会以文字形式说明``正在扫描Wifi''或``扫描成功,正在查询''。
 如果网络出现错误则提示``连接失败''。
 \item 上传位置时,需要输入房间号和当前的X、Y坐标,然后点击上传,应用会在当前位置扫描5组数据,取平均后上传到服务器。整个过程持续20秒左右,有进度条标示进度,并且也会以文本形式
 说明当前已经扫描完多少组数据。
 \item 应用在启动时会显示一个启动画面,之后进入主界面。在主界面中可以通过点击``查询位置''和``上传数据''两个标签或者滑动屏幕来切换功能。标签下方有一个色带,可以标示当前所在
 的功能页面。页面切换时滑动条会以动画的形式滑动到对应的新标签。
\end{enumerate}

\section{实现方法}
  \subsection{服务器端的实现方法}
  \subsection{客户端的实现方法}
  客户端的Java代码实现了四个Activity。
    \subsubsection{SplashScreen}
    ???
    \subsubsection{HeaderActivity}
    HeaderActivity实现了标签的切换功能和相应的动画效果。这个Activity加载了header.xml布局文件,最上面一栏包含两个TextView形式的标签,中间一栏ImageView是展示色带滑动动画的区域,
    最下面一栏是一个android.support.v4.view.ViewPager,用于装载页面的内容,实现滑动效果。
    
    在HeaderActivity中实现了三个内部类。
    \begin{enumerate}[1)]
     \item TagOnClickListener是标签点击事件的监听器,两个标签分别有对应的监听器,在点击后使ViewPager切换到对应页面。
     \item MyPagerAdapter是一个ViewPager配置器类,用于控制ViewPager的行为。
     \item MyOnPageChangeListener用于监听ViewPager中页面的改变,从而生成色带移动的动画,产生滑动效果。
    \end{enumerate}

    \subsubsection{QueryActivity}
    QueryActivity是一个实现位置查询功能的Activity。它加载了query\_main.xml布局文件,包含一个查询按钮,一个默认隐藏的进度条,一个显示查询结果的TextView和一个存放地图的
    默认隐藏ImageView。% 这个类中实现了内部类QueryTask,并定义了方法onCreate,onResume,onPause方法,定义了方法displayMap,onSensorChanged和onAccuracyChanged。
    \begin{enumerate}[1)]
     \item 内部类QueryTask继承了AsyncTask,用于后台执行查询请求。它还有一个内部类WifiReceiver继承了BroadCastReceiver。doInBackground方法中
     实例化了一个WifiReceiver对象,注册它接受WifiManager.SCAN\_RESULTS\_AVAILABLE\_ACTION的广播。之后调用startScan开始扫描Wifi。这个命令是非阻塞性的,因此需要不断sleep,直到
     WifiReceiver接收到扫描完成的信号,并设置相应的全局变量,才继续工作。这个方法中接下来会把扫描结果发送给服务器,然后获取分析结果,最后把房间号以文本形式写出,将算出的坐标标在地图上。
     \item 进度条的展示。QueryTask是以Integer类型记录进度的,具体来说用了两个整数。第一个整数表示已经过去的时间,因为时间主要用于sleep,这个值在每次sleep后更新。第二个整数
     只取0和1两个值,表示wifi扫描是否完成。进度条默认隐藏,因此在启动QueryTask之前先将其设置成可见,然后在onProgressUpdate里面修改,最后在onPostExecute里重新设成隐藏。
    \end{enumerate}

    \subsubsection{InputActivity}
\section{人员分工}
\section{测试结果}
\section{结构图}
\end{CJK*}
\end{document}
